\chapter{File Formats}
The design of the OS for the Z80 project relies on a number of custom file
formats.  These have been designed to be simple but flexible.

\section{Assembly files *.z8a}
All the Z80 assembly code for the project is in files named with a .z8a
extension.  There is no special format to these files, however the use of the
unique file extension is useful in highlighting the code correctly and
distinguishing it from PIC assembly code, for example.

\section{Bootable files *.z8b}
Currently there is only one *.z8b file defined.  This is the BOOT.Z8B default
kernel image.  The basic format for this file type is the same as other
executables (see Section \ref{sec:exes} Executable Files, for details), however 
to qualify as bootables, they must be compiled to run from address \$0000.  The
code will be loaded from this file up to address \$7FFF or the end of the file,
whichever comes first.  If a kernel grows beyond 32KB it is then responsible for
loading the rest of the file into other pages of memory once it has been
bootstrapped and is running itself.  A BIOS variable is left by the bootloader
to indicate which of the mass-storage devices the boot image was found on.  This
is sufficient currently to identify the BOOT.Z8B file to load any further data
after boot.

Other files with this extension are reserved for future implementations of more
flexible bootloaders.

\section{Program files *.z8p}
Standing for Z80 Program, the Z8P file is a standard executable format (see
Section \ref{sec:exes}.  These programs must be compiled to run from address
\$8000.  They will be loaded into memory until they reach address \$FFFF, or the
end of the file, whichever comes first.  If the program is larger than 32KB then
it must request additional memory from the OS and add load these new pages with
the rest of the code/data.

There are several entry points into a Program executable.  These have yet to be
defined but will include interrupt service routines etc.

\section{Library files *.z8l}
Library files are a special instance of executable files.  Similar in purpose to
.dll or .so files on Windows or Linux, they contain routines that may be used in
many different programs that can be called into at run-time.  They are loaded by
the OS in response to library requests from programs.  Only one copy of each
library will ever be loaded at a time so all functions must be stateless i.e.
they have no internal memory so that they can be called from multiple concurrent
threads in interrupts etc.

Library file format details are not yet decided, other than the overall
executable format.

\section{Executable Files}
\label{sec:exes}
All of the executable files share the following basic format:

\noindent
\begin{tabular}{llp{11.5cm}}
 \textbf{Offset}&\textbf{Bytes}&\textbf{Description}\\
 0&1&Version key, for the Mark 2 project this will always be \$02.  This ensures
     that the system does not try and run files for incompatible hardware.\\
 1&2&Header length (HL), a 16 bit count from the start of the file to the first
     byte of the program code.  This is always 3 or more to encompass the
     Version key and the header length itself.  This flexible offset allows for
     any data up to 64KB in size to be stored in the header even if the software
     loading the program does not understand the header.  Uses envisioned
     include author information, some sort of permissions structure or other
     general meta-data.\\
 HL&Any&The actual program code starts in the first byte after the header.\\
\end{tabular}

The intention of using this format is to keep the code simple (read three bytes
then perform a single seek operation to find the data) allowing fast loading of
software.  However there is room for version numbers or any other type of
arbitrary meta data in the header even if the software loading this file doesn't
understand what the header fields actually mean.
