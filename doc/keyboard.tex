% keyboard driver documentation

\section{Keyboard Controller Commands}
You can set various keyboard parameters by writing to the KEY\_DATA port.  The
basic format for these commands is an upper nibble specifiying the command and
some parameters in the lower nibble (allowing a total of 16 commands).

\begin{tabular}{llp{6cm}l}
 \textbf{Command}&\textbf{Name}&\textbf{Description}&\textbf{Default}\\
 1100xxxx&KB\_SET\_TM&Sets the translation mode to xxxx, this is an internal 
                      setting that defines what bytes are sent to the Z80 on
                      keypress. See Table \ref{tab:translators} for details on 
                      implemented translations.& 1 (ASCII)\\
 1101000x&KB\_SET\_REL&If x is 1, all subsequent ``normal keys'' will send
                       release codes (0xF0 followed by keycode) on release.&0\\
 1110000x&KB\_SET\_CMD\_REL&If x is 1, all subsequent control keys will send
                            release codes (0xF0 followed by keycode on release.
 &1\\
 11110xxx&KB\_SET\_LEDS&This command sets the LEDs (and internal state flags)
                        to xxx.  The MSb is CAPS Lock, the middle is NUM Lock
                        and the LSb is SCROLL Lock.&010\\
\end{tabular}

\section{Translation Modes}
In the firmware implementation a relatively abstract lookup mechanism called
translate\_key is included.  This allows arbitrary byte patterns to be sent on
any keypress.  The specifications of these translators is in the following
sections, they can be selected with the KB\_SET\_TM command.  Table
\ref{tab:translators} has a summary list of available translator functions.

\begin{table}
 \begin{tabular}{rlp{7cm}}
  \textbf{Number}&\textbf{Name}&\textbf{Description}\\
  0&None&No translation is done, LEDs are kept up to date and Pause/Break is
         filtered to send only one byte.  F7 will also send 0xF7 instead of 0x83
         other than that all key-press related codes will be sent as Set 2
         scancodes.\\
  1&ASCII&Keycodes that map directly will be sent as ASCII codes these are
          ``normal keys'' which won't send release info by default.  Command
          keys (e.g. shift, ctrl, cursor keys etc.) will send custom ndcodes
          which guarantee 1 byte identification.  They will send release info
          by default.\\
 \end{tabular}
 \caption{Translation Modes}
 \label{tab:translators}
\end{table}

\section{ASCII Mode Codes}
In ASCII Mode, printable characters will be sent, CAPS, SHIFT and NUM are taken
into account before the byte is sent.  In addition keys with direct mapping to
ASCII codes will be sent as ASCII (Del = 127, Backspace = 8, Tab = 9, Return and
Enter = 10).  The only exceptions to this are two (UK) keyboard keys that don't
map to standard ASCII codes, £ and ¬.  The £ symbol is mapped to 0xA3, which 
matches the GPU's internal font mapping, the ¬ symbol has no representation in
the GPU and is replaced with a degrees symbol 0xB0.  Neither of these codes are
used in ndcodes to avoid confusion.

Other keys are sent as ndcodes which is a simple mapping which makes sure bit 7
is set for all control keys and assigns a single unique byte to each key.  These
are shown in the table below.

\subsection{Control Keys}
\begin{tabular}{llp{7cm}}
 \textbf{Mnemonic}&\textbf{Hex}&\textbf{Description}\\
 CC\_ESC&0x80&Escape\\
 CC\_F1&0x81&F1\\
 CC\_F2&0x82&F2\\
 CC\_F3&0x83&F3\\
 CC\_F4&0x84&F4\\
 CC\_F5&0x85&F5\\
 CC\_F6&0x86&F6\\
 CC\_F7&0x87&F7\\
 CC\_F8&0x88&F8\\
 CC\_F9&0x89&F9\\
 CC\_F10&0x8A&F10\\
 CC\_F11&0x8B&F11\\
 CC\_F12&0x8C&F12\\
 CC\_PRINT&0x8D&Print Screen\\
 CC\_SCROLL&0x8E&Scroll Lock\\
 CC\_PAUSE&0x8F&Pause/Break\\
 CC\_NUM&0x90&Num Lock\\
 CC\_CAPS&0x91&Caps Lock\\
 CC\_LSHIFT&0x92&Left Shift\\
 CC\_RSHIFT&0x93&Right Shift\\
 CC\_CTRL&0x94&Left Control\\
 CC\_CTRLR&0x95&Right Control\\
 CC\_ALT&0x96&Left Alt\\
 CC\_ALTR&0x97&Right Alt\\
 CC\_GUIL&0x98&Left GUI (Windows key)\\
 CC\_GUIR&0x99&Right GUI (Windows key)\\
 CC\_APPS&0x9A&Apps (Looks like a menu on most keyboards next to Windows Key\\
 CC\_INS&0x9B&Insert\\
 CC\_HOME&0x9C&Home\\
 CC\_END&0x9D&End\\
 CC\_PGUP&0x9E&Page Up\\
 CC\_PGDN&0x9F&Page Down\\
 CC\_PSF&0xA0&Fake scan code, trapped internally.\\
 CC\_PSR&0xA1&Print Screen\\
 \multicolumn{3}{c}{0xA3 is £ symbol in our character set, don't use it here}\\
 CC\_LEFT&0xA4&Left cursor key\\
 CC\_RIGHT&0xA5&Right cursor key\\
 CC\_UP&0xA6&Up cursor key\\
 CC\_DOWN&0xA7&Down cursor key\\
\end{tabular}

\subsection{Media Keys}
\begin{tabular}{llp{7cm}}
 CC\_MNXT&0xA7&Media next\\
 CC\_MPRV&0xA8&Media previous\\
 CC\_MPP&0xA9&Media play/pause\\
 CC\_MSTP&0xAA&Media stop\\
 CC\_MMT&0xAB&Media mute\\
 CC\_MVU&0xAC&Media volume up\\
 CC\_MVD&0xAD&Media volume down\\
 CC\_MSL&0xAE&Media select\\
 CC\_MEM&0xAF&Media email\\
 \multicolumn{3}{c}{0xB0 is the degrees symbol in our character set}\\
 CC\_MCLC&0xB1&Media calculator\\
 CC\_MCMP&0xB2&Media my computer\\
 CC\_MSRCH&0xB3&Web search\\
 CC\_MHOME&0xB4&Web home\\
 CC\_MBCK&0xB5&Web back\\
 CC\_MFWD&0xB6&Web forward\\
 CC\_MWSP&0xB7&Web stop\\
 CC\_MRFSH&0xB8&Web refresh\\
 CC\_MFV&0xB9&Web favourites\\
\end{tabular}

\subsection{ACPI Control Keys}
\begin{tabular}{llp{7cm}}
 CC\_PWR&0xBA&Power\\
 CC\_SLP&0xBB&Sleep\\
 CC\_WK&0xBC&Wake\\
 
\end{tabular}
