\chapter{PIC BIOS Commands}
After boot, the PIC becomes a slave device on the Z80 I/O bus.  It can then be
interrupted by either a serial command on the debug port from a host PC (see
Chapter \ref{chap:comms}) or by a command from the Z80.

\section{Protocol}
The PIC uses a similar protocol to the Vinculum USB host to throttle access,
there are two flags in the Z80's STATUS register; SD\_TXE and SD\_RXF.  These
two flags permit reading (SD\_RXF) when low and writing (SD\_TXE) when low.  The
PIC will never talk without first being requested for data so SD\_RXF will only
go low when a request has been completed and resulted in some data.  There is a
third bit in the STATUS register called SD\_RDY, this is asserted (low) by the
PIC when an SD card is attached and has been successfully initialised.  Failure
to check the state of this pin before requesting data from the SD card will
almost certainly result in a timeout and garbage data.

\section{Commands}
Commands take the form of single byte values.  There are only 64 commands higher
values are masked so will execute whatever command is specified by the lower six
bits.  If the value is not mentioned below, that command is not used and will
perform no operation.

Commands should be written command and then all bytes of argument in sequence.
No acknowledgement is provded after individual bytes.  If the bytes cause any
processing time to be required the SD\_TXE flag will be set high to prevent
further writes until the PIC is ready for the next byte.  Responses are made as
quickly as possible and are signified by a lowering of SD\_RXF (also SD\_INT).

\subsection{BIOS\_RESET\_CMD \$3F}
This command takes no arguments and returns no data.  On reception of this
command the PIC will immediately do a full system reset (as if the front panel
reset button had been pressed).

\subsection{BIOS\_READ\_VAR\_CMD \$0A}
Takes one byte argument; the address to be read, returns one byte of data; the
contents of requested address in the PIC's internal EEPROM memory.  This is used
to fetch BIOS parameters such as the boot order and system clock speed.

Defined registers:\\
\noindent BIOS\_BOOT\_ADDRESS, Address \$00, contains the BIOS boot order.
\noindent BIOS\_CLOCK\_ADDRESS, Address \$01, indicates the system clock speed.

\subsection{BIOS\_WRITE\_VAR\_CMD \$0B}
Takes two byte arguments; the address followed by the data.  Works in the same
way as BIOS\_READ\_VAR\_CMD to write to the PIC's internal EEPROM memory.  No
response is made on the completion of this command.

\subsection{SD\_CARD\_CID \$20}
Fetch the Card ID info from attached SD card.  This takes no bytes of arguments
and returns 17 bytes of data.  The first byte of data is a status message, the
ascii character O (letter, not zero) indicates the command was successfull.  A
response starting with a letter E indicates an error, in this case, the second
byte of the response is the last command that was sent to the SD card, followed
by the error code returned.  The rest of the packet is padded with undefined
data.  A response begining with the character T indicates there was a timeout
whilst accessing the card.  In this case the second byte is the command last
sent to the SD card followed by the last response received.

\subsection{SD\_CARD\_CSD \$21}
Fetch the Card Specific Data from the SD card (mainly useful for determining the
size and class; HCSD or standard capacity).  Takes no arguments and returns 17
bytes of data.  The return format is as specified above.

\subsection{SD\_CARD\_READ\_BLOCK \$22}
Reads a 512 byte block of data from the SD card.  Takes a 4 byte argument, for
HCSD this is the block number to fetch.  For standard capacity cards it is a
byte address of the block to fetch.  This command returns 515 bytes, the first
is the response status, (okay, error, timeout; see above) the following 512 are
the data and the final pair are the CRC supplied by the SD card.  As with the
previous two commands this command always returns the requested number of bytes
regardless of whether the command fails or not.
